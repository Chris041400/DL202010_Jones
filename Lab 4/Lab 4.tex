% Digital Logic Report Template
% Created: 2020-01-10, John Miller

%==========================================================
%=========== Document Setup  ==============================

% Formatting defined by class file
\documentclass[11pt]{article}

% ---- Document formatting ----
\usepackage[margin=1in]{geometry}	% Narrower margins
\usepackage{booktabs}				% Nice formatting of tables
\usepackage{graphicx}				% Ability to include graphics

%\setlength\parindent{0pt}	% Do not indent first line of paragraphs 
\usepackage[parfill]{parskip}		% Line space b/w paragraphs
%	parfill option prevents last line of pgrph from being fully justified

% Parskip package adds too much space around titles, fix with this
\RequirePackage{titlesec}
\titlespacing\section{0pt}{8pt plus 4pt minus 2pt}{3pt plus 2pt minus 2pt}
\titlespacing\subsection{0pt}{4pt plus 4pt minus 2pt}{-2pt plus 2pt minus 2pt}
\titlespacing\subsubsection{0pt}{2pt plus 4pt minus 2pt}{-6pt plus 2pt minus 2pt}

% ---- Hyperlinks ----
\usepackage[colorlinks=true,urlcolor=blue]{hyperref}	% For URL's. Automatically links internal references.

% ---- Code listings ----
\usepackage{listings} 					% Nice code layout and inclusion
\usepackage[usenames,dvipsnames]{xcolor}	% Colors (needs to be defined before using colors)

% Define custom colors for listings
\definecolor{listinggray}{gray}{0.98}		% Listings background color
\definecolor{rulegray}{gray}{0.7}			% Listings rule/frame color

% Style for Verilog
\lstdefinestyle{Verilog}{
	language=Verilog,					% Verilog
	backgroundcolor=\color{listinggray},	% light gray background
	rulecolor=\color{blue}, 			% blue frame lines
	frame=tb,							% lines above & below
	linewidth=\columnwidth, 			% set line width
	basicstyle=\small\ttfamily,	% basic font style that is used for the code	
	breaklines=true, 					% allow breaking across columns/pages
	tabsize=3,							% set tab size
	commentstyle=\color{gray},	% comments in italic 
	stringstyle=\upshape,				% strings are printed in normal font
	showspaces=false,					% don't underscore spaces
}

% How to use: \Verilog[listing_options]{file}
\newcommand{\Verilog}[2][]{%
	\lstinputlisting[style=Verilog,#1]{#2}
}




%======================================================
%=========== Body  ====================================
\begin{document}

\title{ELC 2137 Lab 5: Subtracting}
\author{CJ Jones, Ashlie Lackey}

\maketitle


\section*{Summary}

The purpose of this experiment was to understand how adders can be manipulated into a subtractor. Using two-full adder circuits, an additional carry in bit, or "mode" bit as called in lab, delegates if the inputs are being added or subtracted. The lab also demonstrated the primary difference between subtracting with a borrow, and adding with a carry. The addition method with the two's complement of the second number would produce a value that is not what is expected of "regular" subtraction and would have to be adjusted as such. In learning how to use the mode bit, correct subtraction was able to take place using our circuit. 


\section*{Q\&A}

\begin{enumerate}
	
	
	\item Why did we use two full adders instead of a half adder and a full adder?
	
	The purpose of using two full adders was to allow for an extra carry bit that provides for the subtraction. If a half and full adder was utilized, essentially the number being subtracted couldn't have a signed bit as part of its two's complement. Therefore, the second full adder replacing the half adder allows for a "negative" number to be added to the first number. 
	
	\item How many input combinations would it take to exhaustively test the adder/subtractor?
	
	16 input combinations would be required to exhuastively test the adder/subtractor. This can be calculated from there being 4 possible cases of each input, and each input needs to get tested against the other four inputs so, 4*4=16.
	
	\item Why  were  the  combinations  given  in  the  truthtable chosen? 
	
	The combinations in the truthtable were picked to show the importance of a carry bit in your calculations. Without regard for the carry bit in the initial table, the results calculted under "Expected Results" were inconsistent with those found in the "Actual Results" because the expected results used addition with a two's complement with no regard for the carry bit. The circuit that was built accounts for the sign bit in the form of the "mode" being carried in to account for subtraction.
	
	\item Do the results from your adder/subtractor match what  you  would  expect  from  theory?   Explain any discrepancies.
	
	The results match what we would expect from theory, however, they differ from the initial "Expected Results". This is because the subtractor circuit accounts for the sign carry bit, while addition produces the two's complement of the result. In theory, the two's complement is sufficient as a result of subtraction using an "addition" method. However, since the circuit cannot interpret this, it is more effective to use the subtractor method to get the exact result and not the theoretical complement.
	
\end{enumerate}





\section*{Results}

The construction of the subtractor circuit is below, along with the wiring schematic and expected results table. The "Expected Results" page documents the difference between subtraction through two's complement addition and borrow subtraction, which our results demonstrates through the "Actual Results" portion of the experiment.
\clearpage
\begin{figure}
	\includegraphics[width=1.0\textwidth]{"Circuit Picture"}
	\caption{Picture of the Circuit Constructed.}
\end{figure}
\clearpage
\begin{figure}
	\includegraphics[width=1.0\textwidth]{"Schematic"}
	\caption{Picture of the Circuit Constructed.}
\end{figure}
\clearpage
\begin{figure}
	\includegraphics[width=1.0\textwidth]{"Demonstration"}
	\caption{Picture of the Circuit Constructed.}
\end{figure}

\clearpage
\section*{Conclusion}
The results of this experiment revealed the necessity of carry bits when trying to build a subtractor circuit. If you chose to use addition with the two's complement method, you would find yourself getting the wrong answer as a result of the signed result. To combat this, the most significant bit was flipped so that the correct answer was obtained as demonstrated in the Expected Results table under "Actual Results". The takeaway from this lab was the demonstration that we can manipulate tools we already have to fix other issues we are trying to solve. In this case, by combining two full adders and adding a mode bit to combat issues with signs and correct output value, we manipulated addition to work also with subtracting numbers.



\end{document}



\end{document}
