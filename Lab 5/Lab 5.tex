% Digital Logic Report Template
% Created: 2020-01-10, John Miller

%==========================================================
%=========== Document Setup  ==============================

% Formatting defined by class file
\documentclass[11pt]{article}

% ---- Document formatting ----
\usepackage[margin=1in]{geometry}	% Narrower margins
\usepackage{booktabs}				% Nice formatting of tables
\usepackage{graphicx}				% Ability to include graphics

%\setlength\parindent{0pt}	% Do not indent first line of paragraphs 
\usepackage[parfill]{parskip}		% Line space b/w paragraphs
%	parfill option prevents last line of pgrph from being fully justified

% Parskip package adds too much space around titles, fix with this
\RequirePackage{titlesec}
\titlespacing\section{0pt}{8pt plus 4pt minus 2pt}{3pt plus 2pt minus 2pt}
\titlespacing\subsection{0pt}{4pt plus 4pt minus 2pt}{-2pt plus 2pt minus 2pt}
\titlespacing\subsubsection{0pt}{2pt plus 4pt minus 2pt}{-6pt plus 2pt minus 2pt}

% ---- Hyperlinks ----
\usepackage[colorlinks=true,urlcolor=blue]{hyperref}	% For URL's. Automatically links internal references.

% ---- Code listings ----
\usepackage{listings} 					% Nice code layout and inclusion
\usepackage[usenames,dvipsnames]{xcolor}	% Colors (needs to be defined before using colors)

% Define custom colors for listings
\definecolor{listinggray}{gray}{0.98}		% Listings background color
\definecolor{rulegray}{gray}{0.7}			% Listings rule/frame color

% Style for Verilog
\lstdefinestyle{Verilog}{
	language=Verilog,					% Verilog
	backgroundcolor=\color{listinggray},	% light gray background
	rulecolor=\color{blue}, 			% blue frame lines
	frame=tb,							% lines above & below
	linewidth=\columnwidth, 			% set line width
	basicstyle=\small\ttfamily,	% basic font style that is used for the code	
	breaklines=true, 					% allow breaking across columns/pages
	tabsize=3,							% set tab size
	commentstyle=\color{gray},	% comments in italic 
	stringstyle=\upshape,				% strings are printed in normal font
	showspaces=false,					% don't underscore spaces
}

% How to use: \Verilog[listing_options]{file}
\newcommand{\Verilog}[2][]{%
	\lstinputlisting[style=Verilog,#1]{#2}
}




%======================================================
%=========== Body  ====================================
\begin{document}

\title{ELC 2137 Lab 5: Subtracting}
\author{CJ Jones, Ashlie Lackey}

\maketitle


\section*{Summary}

Here is where my summary will go lalala


\section*{Q\&A}

\begin{enumerate}
	\item What is one thing that you still don’t understand about Verilog?
	
	The most confusing this about Verilog is...
\end{enumerate}




\section*{Results}

The results will be discussed here with an inclusion of the screenshots...
\begin{figure}[ht]\centering
	\begin{tabular}{l|rrrr}
		Time (ns): & 0 & 10 & 20 & 30 \\
		\midrule 
		a & 0 & 1 & 0 & 1 \\
		b & 0 & 0 & 1 & 1 \\
		\midrule
		c & 0 & 0 & 0 & 1 \\
		s & 0 & 1 & 1 & 0 \\ \bottomrule
	\end{tabular}\medskip
	
\end{figure}
\begin{figure}
	\includegraphics[width=1.0\textwidth]{"HalfAdder"}
	\caption{Sin Wave of the Half Adder}
\end{figure}
\clearpage

\begin{figure}[ht]\centering
	\begin{tabular}{l|rrrrrrrr}
		Time (ns): & 0 & 10 & 20 & 30 & 40 & 50 & 60 & 70 \\
		\midrule 
		a &  0 & 1 & 0 & 1 & 0 & 1 & 0 & 1 \\
		b & 0 & 0 & 1 & 1 & 0 & 0 & 1 & 1\\
		cin & 0 & 0 & 0 & 0 & 1 & 1 & 1 & 1 \\
		\midrule
		c & 0 & 0 & 0 & 1 & 0 & 1 & 1 & 1 \\
		s & 0 & 1 & 1 & 0 & 1 & 0 & 0 & 1 \\ \bottomrule
	\end{tabular}\medskip
	

\end{figure}
\begin{figure}
	\includegraphics[width=1.0\textwidth]{"FullAdder"}
	\caption{Sin Wave of the Full Adder}
\end{figure}
\clearpage

\begin{figure}[ht]\centering
	\begin{tabular}{l|rrrrrrrr}
		Time (ns): & 0 & 10 & 20 & 30 & 40 & 50 & 60 & 70 \\
		\midrule 
		a &  0 & 1 & 0 & 1 & 0 & 1 & 0 & 1 \\
		b & 0 & 0 & 1 & 1 & 0 & 0 & 1 & 1\\
		cin & 0 & 0 & 0 & 0 & 1 & 1 & 1 & 1 \\
		\midrule
		c & 0 & 0 & 0 & 1 & 0 & 1 & 1 & 1 \\
		s & 0 & 1 & 1 & 0 & 1 & 0 & 0 & 1 \\ \bottomrule
	\end{tabular}\medskip
	

\end{figure}
\begin{figure}
	\includegraphics[width=1.0\textwidth]{"AddSubReal"}
	\caption{Sin Wave of the Addition Subtraction }
\end{figure}
\clearpage

The results match what was expected as the screenshotted waveform results confirm the expected results table. 
\section*{Code}
\begin{lstlisting}[style=Verilog,caption=Half Adder Design Code,label=code:ex ]
// Ashlie Lackey and CJ Jones , ELC 2137 , 2020 -2 -13
module halfadder ( input a , b ,
output s , c ) ;
// When using primitive instances , output is always the first port
xor (s , a , b ) ;
and (c , a , b ) ;
endmodule // halfadder

\end{lstlisting}

\begin{lstlisting}[style=Verilog,caption=Half Adder Testbench Code,label=code:ex ]
module example_and(
input a, b,
output c);
// c = a AND b
and(c,a,b);
endmodule
\end{lstlisting}

\begin{lstlisting}[style=Verilog,caption=Full Adder Design Code,label=code:ex ]
// Ashlie Lackey and CJ Jones , ELC 2137 , 2020 -2 -13
module fulladder (
input ain , bin , cin ,
output cout , sout
) ;

// Internal signals
wire c1 , c2 , s1 ;
// One instance of halfadder
halfadder ha0 (
. a ( ain ) , . b ( bin ) ,
. c ( c1 ) , . s ( s1 )
) ;
// second halfadder
halfadder ha1 (
. a ( s1 ) , . b ( cin ) ,
. c ( c2 ) , . s ( sout )
) ;

// Define carry output
xor (cout , c1 , c2 ) ;
endmodule // fulladder

\end{lstlisting}

\begin{lstlisting}[style=Verilog,caption=Full Adder Testbench Code,label=code:ex ]
// Ashlie Lackey and CJ Jones , ELC 2137 , 2020 -2 -13
module fulladder_test () ;
reg cin_t , a_t , b_t ;
wire cout_t , s_t ;
//instantiate fulladder
fulladder FA (. ain ( a_t ) , . bin ( b_t ) , . cin ( cin_t ) , . sout ( s_t ), . cout (cout_t) ) ;
// module here

initial begin
cin_t =0; a_t =0; b_t =0; #10;
//case2
{ a_t , b_t, cin_t } = 3'b100 ; #10;
//case3
{ a_t , b_t, cin_t } = 3'b010 ; #10;
//case4
{ a_t , b_t, cin_t } = 3'b110 ; #10;
//case5
{ a_t , b_t, cin_t } = 3'b001 ; #10;
//case6
{ a_t , b_t, cin_t } = 3'b101 ; #10;
//case7
{ a_t , b_t, cin_t } = 3'b011 ; #10;
//case8
{ a_t , b_t, cin_t } = 3'b111 ; #10;

$finish ;
end

endmodule // fulladder_test

\end{lstlisting}

\begin{lstlisting}[style=Verilog,caption=2-bit Adder Design Code,label=code:ex ]
// Ashlie and CJ , ELC 2137 , 2020 -2 -13
module addsub (
input [1:0] a , b ,
input mode ,
output [1:0] sum ,
output cbout
) ;
// Internal signals
wire c1 , c2 ;
wire [1:0] b_n ;
// Invert b input for subtraction
assign b_n[0] = b[0] ^ mode;
assign b_n[1] = b[1] ^ mode;
//first full adder
fulladder fa0 (
. ain ( a [0]) , . bin ( b_n [0]) , . cin (
mode) ,
. cout ( c1 ) , . sout ( sum [0])
) ;
//second full adder
fulladder fa1 (
. ain ( a [1]) , . bin ( b_n [1]) , . cin (
c1 ) ,
. cout ( c2 ) , . sout ( sum [1])
) ;
// Convert carry to borrow when
// subtracting
assign cbout = c2 ^ mode;
endmodule // addsub

\end{lstlisting}

\begin{lstlisting}[style=Verilog,caption=2-bit Adder Testbench Code,label=code:ex ]
// Ashlie Lackey , ELC 2137 , 2020 -2 -19
module addsub_test () ;
reg m_t;
reg [1:0] a_t, b_t;
wire cout_t ;
wire [1:0] s_t ;
// instantiate adder/subtractor
addsub AS(. a ( a_t ) , . b ( b_t ) , . mode ( m_t ) , . sum ( s_t ), . cbout (cout_t) ) ;
// module here

initial begin
//Subtraction Cases
//case1
{ a_t[1], b_t[1], a_t[0], b_t[0], m_t } = 5'b00001 ; #10;
//case2
{ a_t[1], b_t[1], a_t[0], b_t[0], m_t } = 5'b00011 ; #10;
//case3
{ a_t[1], b_t[1], a_t[0], b_t[0], m_t } = 5'b01001 ; #10;
//case4
{ a_t[1], b_t[1], a_t[0], b_t[0], m_t } = 5'b01011 ; #10;
//case5
{ a_t[1], b_t[1], a_t[0], b_t[0], m_t } = 5'b00111 ; #10;
//case6
{ a_t[1], b_t[1], a_t[0], b_t[0], m_t } = 5'b10011 ; #10;
//case7
{ a_t[1], b_t[1], a_t[0], b_t[0], m_t } = 5'b10001 ; #10;



//Addition Cases
//case8
{ a_t[1], b_t[1], a_t[0], b_t[0], m_t } = 5'b00000 ; #10;
//case9
{ a_t[1], b_t[1], a_t[0], b_t[0], m_t } = 5'b00010 ; #10;
//case10
{ a_t[1], b_t[1], a_t[0], b_t[0], m_t } = 5'b01000 ; #10;
//case11
{ a_t[1], b_t[1], a_t[0], b_t[0], m_t } = 5'b01010 ; #10;
//case12
{ a_t[1], b_t[1], a_t[0], b_t[0], m_t } = 5'b00110 ; #10;
//case13
{ a_t[1], b_t[1], a_t[0], b_t[0], m_t } = 5'b10010 ; #10;
//case14
{ a_t[1], b_t[1], a_t[0], b_t[0], m_t } = 5'b10000 ; #10;

$finish ;
end

endmodule // addsub_test


\end{lstlisting}



\end{document}



\end{document}
