% Digital Logic Lab Report 2
% Created: 2020-01-27, CJ Jones and Ashley Lackey

%==========================================================
%=========== Document Setup  ==============================

% Formatting defined by class file
\documentclass[11pt]{article}

% ---- Document formatting ----
\usepackage[margin=1in]{geometry}	% Narrower margins
\usepackage{booktabs}				% Nice formatting of tables
\usepackage{graphicx}				% Ability to include graphics

%\setlength\parindent{0pt}	% Do not indent first line of paragraphs 
\usepackage[parfill]{parskip}		% Line space b/w paragraphs
%	parfill option prevents last line of pgrph from being fully justified

% Parskip package adds too much space around titles, fix with this
\RequirePackage{titlesec}
\titlespacing\section{0pt}{8pt plus 4pt minus 2pt}{3pt plus 2pt minus 2pt}
\titlespacing\subsection{0pt}{4pt plus 4pt minus 2pt}{-2pt plus 2pt minus 2pt}
\titlespacing\subsubsection{0pt}{2pt plus 4pt minus 2pt}{-6pt plus 2pt minus 2pt}

% ---- Hyperlinks ----
\usepackage[colorlinks=true,urlcolor=blue]{hyperref}	% For URL's. Automatically links internal references.

% ---- Code listings ----
\usepackage{listings} 					% Nice code layout and inclusion
\usepackage[usenames,dvipsnames]{xcolor}	% Colors (needs to be defined before using colors)

% Define custom colors for listings
\definecolor{listinggray}{gray}{0.98}		% Listings background color
\definecolor{rulegray}{gray}{0.7}			% Listings rule/frame color

% Style for Verilog
\lstdefinestyle{Verilog}{
	language=Verilog,					% Verilog
	backgroundcolor=\color{listinggray},	% light gray background
	rulecolor=\color{blue}, 			% blue frame lines
	frame=tb,							% lines above & below
	linewidth=\columnwidth, 			% set line width
	basicstyle=\small\ttfamily,	% basic font style that is used for the code	
	breaklines=true, 					% allow breaking across columns/pages
	tabsize=3,							% set tab size
	commentstyle=\color{gray},	% comments in italic 
	stringstyle=\upshape,				% strings are printed in normal font
	showspaces=false,					% don't underscore spaces
}

% How to use: \Verilog[listing_options]{file}
\newcommand{\Verilog}[2][]{%
	\lstinputlisting[style=Verilog,#1]{#2}
}




%======================================================
%=========== Body  ====================================
\begin{document}

\title{ELC 2137 Lab 3: Adders }
\author{Ashlie Lackey, CJ Jones}

\maketitle


\section*{Summary}

The purpose of this experiment was to learn how to build different kinds of adders. By examining the truth tables to half, full, and two-bit adders, occasionally with their decimal equivalent, it was easier to examine what types of gate combinations were necessary for each type of adder. For example, with the half-adder, it becomes apparent that an XOR gate is necessary to get the correct value of the sum (S), while the carry (C) utilizes an AND gate to correctly add the binary digits. This also formed the basis for forming the more complicated full-adder and two-bit adder circuits. Additionally, students used schematics and wiring diagrams to aid in the construction of the adders.


\section*{Q\&A}

\begin{enumerate}
	\item Which gates could we use for combining the carry bits?
	
	To combine the carry bits we could use the AND or XOR gate. Normally an AND gate is preferable for finding the carry bit itself from an input, but the XOR is easier for combining two carry bits into a higher carry bit. 
	
	\item Which one should we use and why?
	
	We should use the XOR gate because it combines both the carry bits but only when exclusively one is high. That way all cases of the bits only being high one at a time are accounted for, and not when they both are high. 
\end{enumerate}




\section*{Results}

The included documents below demonstrate the construction of each of the three adders being built along with the corresponding truth table outputs they produce. 

\begin{figure}
	\includegraphics [width=1.0\textwidth]{"HALFADDER"}
	\caption{Half Adder Circuit}

\end{figure}
\begin{figure}[ht]\centering
	\begin{tabular}{c|c||c|c||c}
		\toprule
		A & B & C & S & Decimal \\
		\midrule
		0 & 0 & 0 & 0 & 0 \\
		0 & 1 & 0 & 1 & 1 \\
		1 & 0 & 0 & 1 & 1 \\
		1 & 1 & 1 & 0 & 2 \\
		\bottomrule
	\end{tabular} 
	
	\caption{Half adder truth table}
	
\end{figure}

\begin{figure}
\includegraphics[width=1.0\textwidth]{"FUULADDER"}
\caption{Full Adder Circuit}
\end{figure}
\begin{figure}[ht]\centering
	\begin{tabular}{c|c|c||c|c||c}
		\toprule
		$C_{m}$ & A & B & $C_{out}$ & S & Decimal \\
		\midrule
		0 & 0 & 0 & 0 & 0 & 0  \\
		0 & 0 & 1 & 0 & 1 & 1  \\
		0 & 1 & 0 & 0 & 1 & 1  \\
		0 & 1 & 1 & 1 & 0 & 2  \\
		1 & 0 & 0 & 0 & 1 & 1  \\
		1 & 0 & 1 & 1 & 0 & 2  \\
		1 & 1 & 0 & 1 & 0 & 2  \\
		1 & 1 & 1 & 1 & 1 & 3  \\
		\bottomrule
	\end{tabular} 
	
	\caption{Full Adder Truth Table}
	
\end{figure}


\begin{figure}
	\includegraphics[width=1.0\textwidth]{"TWO-BITADDER"}
	\caption{Two-Bit Adder Circuit}
\end{figure}

\begin{figure}[ht]\centering
	\begin{tabular}{c|c|c||c}
		\toprule
		$C_{in}$ & $A_{2}$$A_{1}$ & $B_{2}$$B_{1}$ &$C_{out}$$S_{2}$$S_{1}$  \\
		\midrule
		0 & 00 & 00 & 0 0 0\\
		0 & 00 & 01 & 0 0 1\\
		0 & 01 & 01 & 0 1 0\\
		0 & 10 & 01 & 0 1 1\\
		0 & 10 & 10 & 1 0 0\\
		1 & 10 & 10 & 1 0 1\\
		1 & 11 & 11 & 1 1 1\\
		\bottomrule
	\end{tabular} 
	
	\caption{2-Bit Adder Truth Table}
	
\end{figure}



\clearpage

\section*{Conclusion}

This lab demonstrated the ability to build circuits that have additive capabilities using binary digits. While the circuits built did not have the aesthetic appeal and color coded wires necessary for building more complex circuits, not having the exact number and types of wires hindered this ability. Therefore, given the tools in lab, functioning circuits were built, despite not being the prettiest to look at. A useful skill gained from this lab is utilizing the ordering of switches and LEDs to correspond with the truth tables in order to test the circuit with as much ease as possible. As can be noted, in the two-bit adder circuit, labels using paper were utilized to keep track of this order, again to make testing easier. Overall, this lab demonstrated how to build adder circuits. Being combining similar structures, although with varying carry-ins are sources for input and output, the AND and XOR gates were sufficient to build a half-, full-, and 2-bit adders.




\end{document}
