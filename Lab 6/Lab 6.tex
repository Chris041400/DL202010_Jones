% Digital Logic Report Template
% Created: 2020-01-10, John Miller

%==========================================================
%=========== Document Setup  ==============================

% Formatting defined by class file
\documentclass[11pt]{article}

% ---- Document formatting ----
\usepackage[margin=1in]{geometry}	% Narrower margins
\usepackage{booktabs}				% Nice formatting of tables
\usepackage{graphicx}				% Ability to include graphics

%\setlength\parindent{0pt}	% Do not indent first line of paragraphs 
\usepackage[parfill]{parskip}		% Line space b/w paragraphs
%	parfill option prevents last line of pgrph from being fully justified

% Parskip package adds too much space around titles, fix with this
\RequirePackage{titlesec}
\titlespacing\section{0pt}{8pt plus 4pt minus 2pt}{3pt plus 2pt minus 2pt}
\titlespacing\subsection{0pt}{4pt plus 4pt minus 2pt}{-2pt plus 2pt minus 2pt}
\titlespacing\subsubsection{0pt}{2pt plus 4pt minus 2pt}{-6pt plus 2pt minus 2pt}

% ---- Hyperlinks ----
\usepackage[colorlinks=true,urlcolor=blue]{hyperref}	% For URL's. Automatically links internal references.

% ---- Code listings ----
\usepackage{listings} 					% Nice code layout and inclusion
\usepackage[usenames,dvipsnames]{xcolor}	% Colors (needs to be defined before using colors)

% Define custom colors for listings
\definecolor{listinggray}{gray}{0.98}		% Listings background color
\definecolor{rulegray}{gray}{0.7}			% Listings rule/frame color

% Style for Verilog
\lstdefinestyle{Verilog}{
	language=Verilog,					% Verilog
	backgroundcolor=\color{listinggray},	% light gray background
	rulecolor=\color{blue}, 			% blue frame lines
	frame=tb,							% lines above & below
	linewidth=\columnwidth, 			% set line width
	basicstyle=\small\ttfamily,	% basic font style that is used for the code	
	breaklines=true, 					% allow breaking across columns/pages
	tabsize=3,							% set tab size
	commentstyle=\color{gray},	% comments in italic 
	stringstyle=\upshape,				% strings are printed in normal font
	showspaces=false,					% don't underscore spaces
}

% How to use: \Verilog[listing_options]{file}
\newcommand{\Verilog}[2][]{%
	\lstinputlisting[style=Verilog,#1]{#2}
}




%======================================================
%=========== Body  ====================================
\begin{document}

\title{ELC 2137 Lab 6: MUX and 7-segment Decoder}
\author{CJ Jones, Ashlie Lackey}

\maketitle


\section*{Summary}

This lab explored using a Basys3 board to produce an 8-bit number on a 7-segment display through a MUX combinational logic design. The MUX used had two inputs that were four bits each. Using Verilog, some skills gained in this lab include: writing a multiplexer utilizing the conditional operator, using always block, using multi-bit signals, producing a toplevel module, using constraint files,and creating a design on a FPGA board. Overall, this lab demonstrated how to utilize software and programmable logic to produce a hardware output.


\section*{Q\&A}

\begin{enumerate}
	\item How many wires are connected to the 7-segment display?
	
	In our version of the seven segment display, only 1 wire is connected to the 7-segment display
	
	\item   If the segments were not all connected together, how many wires would there have to be?
	
	\item Why do we prefer the current method vs. separating all of the segments?
	
\end{enumerate}




\section*{Results}

Below are the results for running the half-, full-, and 2-bit adders. The design and testbench code for each adder is included along with each expected results table and the block diagram of each digital circuit. As the ERT's and screenshots each indicate, the results from the testbench simulations do match what is expected.


\clearpage

The results match what was expected as the screenshotted waveform results confirm the expected results table. 


\section*{Code}
\begin{lstlisting}[style=Verilog,caption=mux2-4b Code,label=code:ex ]
// Ashlie Lackey and Chris Jones , ELC 2137, 2020 -2-26
module mux2_4b(
input [3:0]in0,
input [3:0]in1,
input sel,
output [3:0]out
);

assign out = sel?in1:in0;
endmodule
\end{lstlisting}

\begin{lstlisting}[style=Verilog,caption=MUX Testbench Code,label=code:ex ]
// Ashlie Lackey and Chris Jones , ELC 2137, 2020 -2-26
module mux2_4b_test();
reg [3:0]in0_t, in1_t;
reg sel_t;
wire [3:0]out_t;

mux2_4b Mux(.in1(in1_t), .in0(in0_t), .sel(sel_t), .out(out_t));

initial begin
in0_t = 4'b0000;
in1_t = 4'b0001;
sel_t = 0;
#10;

in0_t = 4'b0001;
in1_t = 4'b0010;
sel_t = 1;
#10;

in0_t = 4'b0010;
in1_t = 4'b0001;
sel_t = 0;
#10;

in0_t = 4'b0010;
in1_t = 4'b0001;
sel_t = 1;
#10;

$finish ;
end
endmodule   //mux2_4b_test

\end{lstlisting}


\end{document}



\end{document}
